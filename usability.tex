\documentclass[submission,copyright,creativecommons]{eptcs}
\providecommand{\event}{TFPIE 2016} % Name of the event you are submitting to
\usepackage{url}
\usepackage{color}

\newcommand{\allcomments}[1]{{#1}}
%\newcommand{\allcomments}[1]{}

%% Elena's favorite green (thanks, Fernando!)
\definecolor{ForestGreen}{RGB}{34,139,34}
% Uncomment this if you don't want to show comments
\newcommand{\emcomment}[1]{{\bf \textcolor{ForestGreen}{\allcomments{{#1}}}}}
\newcommand{\todo}[1]{{\bf \color{magenta}{\allcomments{ To-do: {#1}}}}}


\title{TBD: Usability of Clojure error messages (extended abstract).}
\author{Elena Machkasova 
\institute{University of Minnesota, Morris
\email{elenam@morris.umn.edu}}
\and
Henry Fellows
\institute{University of Minnesota, Morris
\email{?@morris.umn.edu}}
\and 
Thomas Hagen
\institute{University of Minnesota, Morris
\email{?@morris.umn.edu}}
\and Sean Stockholm
\institute{University of Minnesota, Morris
\email{?@morris.umn.edu}}
}
\def\titlerunning{Clojure error messages}
\def\authorrunning{E. Machkasova, H. Fellows, T. Hagen \& S. Stockholm }
\begin{document}
\maketitle

\begin{abstract}
TBD
\end{abstract}

\section{Introduction}\label{sec:intro}

\section{Towards teaching Clojure in an introductory class}\label{sec:project}
Languages in the Lisp family have along history of being a successful first programming language for college-level students. \emcomment{cite Felleisen}
They focus on functional abstraction, modularity, generalization, and recursion -- concepts that are essential for computer science methods of 
 problem solving and a structural, systematic approach to software development. 
Functional languages tend to have simpler, more uniform syntax and semantics that students master quickly.

\subsection{Overview of Clojure}\label{subsec:clojure}

\subsection{Challenges}\label{subsec:challenges}

\emcomment{mention IDE}

\section{Error messages adapted for beginners}\label{sec:errors-work}
\emcomment{Make sure that examples are more involved than in the previous paper}



\emcomment{Stack filtering}

\section{Design of usability study}\label{sec:study}

\subsection{Conclusions and future work}\label{sec:future}

\bibliographystyle{eptcs}
\bibliography{usability}
\end{document}



