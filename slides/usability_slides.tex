\documentclass{beamer}
\usepackage{color} %For Comments
\usepackage{beamerthemeshadow}
\mode<presentation>
{
\usetheme{Montpellier}
\usecolortheme{dolphin}
%%% set style for ovelays: lists (and other text) appearing one item at a time
%%% This will create a dimmed preview of next item:
%\setbeamercovered{transparent}
%%% This will hide it entirely:
\setbeamercovered{invisible}
}

%% Elena's favorite green (thanks, Fernando!)
\definecolor{ForestGreen}{RGB}{34,139,34}
\definecolor{BlueViolet}{RGB}{138,43,226}
\definecolor{Coquelicot}{RGB}{255, 56, 0}
\definecolor{Teal}{RGB}{2,132,130}
% Uncomment this if you want to show work-in-progress comments
\newcommand{\comment}[1]{{\bf \tt  {#1}}}
% Uncomment this if you don't want to show comments
%\newcommand{\comment}[1]{}
\newcommand{\emcomment}[1]{\textcolor{ForestGreen}{\comment{Elena: {#1}}}}
\newcommand{\todo}[1]{\textcolor{blue}{\comment{To Do: {#1}}}}
\newcommand{\hfcomment}[1]{\textcolor{Teal}{\comment{Henry: {#1}}}}
\newcommand{\thcomment}[1]{\textcolor{Coquelicot}{\comment{Thomas: {#1}}}}
%%%%%%%%%%%%%%%%%%%%%%%%%%%%%%%%%%%%%%%%%%

\begin{document}
\title{Usability of beginner-oriented Clojure error messages}
\author{Henry Fellows, Thomas Hagen, Sean Stockholm, 
and Elena Machkasova}
\institute[UMM] % (optional, but mostly needed)
{
 % \inst{1}%
  University of Minnesota, Morris
}
\date{5th International Workshop on Trends in Functional Programming in Education, \\
College Park, MD, June 7, 2016}

\begin{frame}
\titlepage
\end{frame}
%frame

\begin{frame}
\frametitle{Table of contents}
\tableofcontents  
\end{frame}

\section{Clojure error messages}

\begin{frame}
\frametitle{Clojure and goals of our project }
Clojure:
\begin{itemize}
\item A Lisp, runs on the JVM
\item Data immutable; state is handled explicitly 
\item Support for concurrency 
\item Rich collection of data structures (lists, vectors, hashmaps, sets,...)
\item Developed by Rich Hickey, released in 2007
\item Has a large community of users and developers 
\end{itemize}

A project at university from Minnesota, Morris (UMM) to make it possible to use Clojure as the first language for teaching beginner students. 
\end{frame}

\begin{frame}
\frametitle{UMM course structure}
UMM current course structure:
\begin{itemize}
\item Introductory: Racket or Python 
\item Second programming course: Data Structures in Java
\item Upper level classes use a variety of languages:  Javascript (web development), C, Ruby (operating systems),  R, Clojure (data processing). 
\end{itemize} 
Seeing a functional language first is helpful (functional abstraction, higher-order functions). 

Students have no problem picking up new languages. 
\end{frame}

\begin{frame}
\frametitle{Clojure in introductory class}
Clojure is a promising language for an introductory class:
\begin{itemize}
\item Functional language
\item Provides a large set of data structures (hashmaps, sets)
 \item Has a large community of developers and users: libraries, participation in open source projects, parallel data processing,  etc. 
\end{itemize} 
\end{frame}

\begin{frame}
\frametitle{Challenges for using Clojure in introductory class}
Challenges:
\begin{itemize}
\item Beginner-friendly IDEs are still under development
\item Syntax is very flexible, so errors are detected late. Example: \\ 
{(first '())} returns {\tt nil}, may cause null pointer exception later.  
\item Errors are Java exceptions, are phrased in terms of Java types:
\emcomment{Example}
\end{itemize}
\end{frame}

\begin{frame}
\frametitle{ Clojure error messages}
\emcomment{Show before/after; two types of error recording: preconditions, Clojure expression processing}
\emcomment{It's not as simple as surrounding the code with try/catch}
\emcomment{Mention somewhere that we use Racket in one of our intro classes; use Java in Data Structures, use some Clojure later}
\end{frame}

\begin{frame}
\frametitle{ Clojure error messages (cont)}
\emcomment{Show before/after; two types of error recording: preconditions (printing the arguments), Clojure expression processing}
\end{frame}

\begin{frame}
\frametitle{ Clojure IDEs}
\emcomment{Difficulties in integration, this is not a current goal}
\end{frame}

\section{Usability study}

\begin{frame}
\frametitle{Usability study}
\emcomment{We have a large student population that has had Racket class}
\emcomment{Study structure}
\end{frame}

\begin{frame}
\frametitle{ Differences between Clojure and Racket}
\emcomment{Key syntax differences, what subset we chose}
\end{frame}

\begin{frame}
\frametitle{Study questions}

\end{frame}

\begin{frame}
\frametitle{Study questions (cont)}

\end{frame}

\section{Results}

\begin{frame}
\frametitle{Study subjects overview}
\emcomment{Stats}
\end{frame}

\begin{frame}
\frametitle{Results}
\emcomment{Results: questions solved and times}
\end{frame}

\begin{frame}
\frametitle{Results: being on track}
\emcomment{Results: questions solved and time}
\end{frame}

\begin{frame}
\frametitle{Results:interview questions}
\emcomment{Interview questions and feedback}
\end{frame}

\section{Conclusions and future work}

\frametitle{Conclusions}
\begin{frame}
\emcomment{Our error messages vs standard}
\emcomment{Comparison to Racket}
\emcomment{Correlation to background}
\end{frame}

\begin{frame}
\frametitle{Conclusions}
\emcomment{Thoughts on study setup}
\end{frame}

\begin{frame}
\frametitle{Future work}
\emcomment{How much do people use error messages?}
\end{frame}

\begin{frame}
\frametitle{Acknowledgments}
	Our research was sponsored by:
	\begin{itemize}
	\item HHMI, UMN UROP, LSAMP
        \item Coginitect, Inc providing funding for participants' compensation 
	\end{itemize}
	{\centering
	\noindent
	Thank you! \par
	Any questions? \par
	}
\end{frame}
\end{document}
