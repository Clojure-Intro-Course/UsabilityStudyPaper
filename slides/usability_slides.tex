\documentclass{beamer}
\usepackage{color} %For Comments
\usepackage{beamerthemeshadow}
\mode<presentation>
{
\usetheme{Montpellier}
\usecolortheme{dolphin}
%%% set style for ovelays: lists (and other text) appearing one item at a time
%%% This will create a dimmed preview of next item:
%\setbeamercovered{transparent}
%%% This will hide it entirely:
\setbeamercovered{invisible}
}

%% Elena's favorite green (thanks, Fernando!)
\definecolor{ForestGreen}{RGB}{34,139,34}
\definecolor{BlueViolet}{RGB}{138,43,226}
\definecolor{Coquelicot}{RGB}{255, 56, 0}
\definecolor{Teal}{RGB}{2,132,130}
% Uncomment this if you want to show work-in-progress comments
\newcommand{\comment}[1]{{\bf \tt  {#1}}}
% Uncomment this if you don't want to show comments
%\newcommand{\comment}[1]{}
\newcommand{\emcomment}[1]{\textcolor{ForestGreen}{\comment{Elena: {#1}}}}
\newcommand{\todo}[1]{\textcolor{blue}{\comment{To Do: {#1}}}}
\newcommand{\hfcomment}[1]{\textcolor{Teal}{\comment{Henry: {#1}}}}
\newcommand{\thcomment}[1]{\textcolor{Coquelicot}{\comment{Thomas: {#1}}}}
\newcommand{\R}[1]{\textcolor{blue}{\bf {#1}}}
\newcommand{\Cl}[1]{\textcolor{ForestGreen}{\bf {#1}}}
%%%%%%%%%%%%%%%%%%%%%%%%%%%%%%%%%%%%%%%%%%

\begin{document}
\title{Usability of beginner-oriented Clojure error messages}
\author{Henry Fellows, Thomas Hagen, Sean Stockholm, 
and Elena Machkasova}
\institute[UMM] % (optional, but mostly needed)
{
 % \inst{1}%
  University of Minnesota, Morris
}
\date{5th International Workshop on Trends in Functional Programming in Education, \\
College Park, MD, June 7, 2016}

\begin{frame}
\titlepage
\end{frame}
%frame

\begin{frame}
\frametitle{Table of contents}
\tableofcontents  
\end{frame}

\section{Overview}

\begin{frame}
\frametitle{Clojure and goals of our project }
Clojure:
\begin{itemize}
\item A Lisp, runs on the JVM
\item Data immutable; state is handled explicitly 
\item Support for concurrency 
\item Rich collection of data structures (lists, vectors, hashmaps, sets,...)
\item Developed by Rich Hickey, released in 2007
\item Has a large community of users and developers 
\end{itemize}

A project at university from Minnesota, Morris (UMM) to make it possible to use Clojure as the first language for teaching beginner students. 
\end{frame}

\begin{frame}
\frametitle{UMM course structure}
UMM current course structure:
\begin{itemize}
\item Introductory: Racket or Python 
\item Second programming course: Data Structures in Java
\item Upper level classes use a variety of languages:  Javascript (web development), C, Ruby (operating systems),  R, Clojure (data processing). 
\end{itemize} 
Seeing a functional language first is helpful (functional abstraction, higher-order functions). 

Students have no problem picking up new languages. 
\end{frame}

\begin{frame}
\frametitle{Clojure in introductory class}
Clojure is a promising language for an introductory class:
\begin{itemize}
\item Functional language
\item Provides a large set of data structures (hashmaps, sets)
 \item Has a large community of developers and users: libraries, participation in open source projects, parallel data processing,  etc. 
\end{itemize} 
\end{frame}

\begin{frame}[fragile]
\frametitle{Challenges for using Clojure in introductory class}
Challenges:
\begin{itemize}
\item Beginner-friendly IDEs are still under development
\item Syntax is very flexible, so errors are detected late. Example: \\ 
{(first '())} returns {\tt nil}, may cause null pointer exception later.  
\item Errors are Java exceptions, are phrased in terms of Java types:
\begin{verbatim}
>(+ "hi")
java.lang.ClassCastException: 
Cannot cast java.lang.String to java.lang.Number
\end{verbatim}
\end{itemize}
Our work: providing beginner-friendly error messages for Clojure. 
\end{frame}

\begin{frame}
\frametitle{ Our project}
To develop materials and setup that would make it possible to use Clojure in an introductory class. 
\begin{itemize}
\item Modifying error messages
\item Evaluating how well new error messages work: usability study comparing our error messages to standard ones and to Racket
\item Future work: integrate our messages into an IDE 
\item Other work: graphical library that abstracts over state and objects; developing lecture notes 
\end{itemize}
\end{frame}

\begin{frame}
\frametitle{ Our project}
To develop materials and setup that would make it possible to use Clojure in an introductory class. 
\begin{itemize}
\item {\bf Modifying error messages}
\item {\bf Evaluating how well new error messages work: usability study comparing our error messages to standard ones and to Racket}
\item Future work: integrate our messages into an IDE
\item Other work: graphical library that abstracts over state and objects; developing lecture notes 
\end{itemize}
\end{frame}

\section{Modifying error messages}

\begin{frame}
\frametitle{ Clojure error messages}
\emcomment{Show before/after; two types of error recording: preconditions, Clojure expression processing}
\emcomment{It's not as simple as surrounding the code with try/catch}
\emcomment{Mention somewhere that we use Racket in one of our intro classes; use Java in Data Structures, use some Clojure later}
\end{frame}

\begin{frame}
\frametitle{ Clojure error messages (cont)}
\emcomment{Show before/after; two types of error recording: preconditions (printing the arguments), Clojure expression processing}
\end{frame}


\section{Usability study}

\begin{frame}
\frametitle{Usability study}
UMM has a large number of students familiar with Racket. 

Study:
\begin{itemize}
\item 16 program fragments with errors, each has Racket and Clojure versions 
\item Divided into 4 levels by difficulty
\item Each participant gets 2 Racket, 2 Clojure questions at each level
\item Overview of Racket, 8 Racket questions
\item Overview of Clojure, 8 Clojure questions
\item Screen capture while solving questions (21 minutes each set)
\item A short interview at the end
\end{itemize}
\end{frame}

\begin{frame}
\frametitle{ Differences between Clojure and Racket}
\emcomment{Key syntax differences, what subset we chose}
\end{frame}

\begin{frame}
\frametitle{ Differences in testing environments}
\emcomment{Would be good to include pictures}
\end{frame}

\begin{frame}
\frametitle{Study questions (example 1)}

\end{frame}

\begin{frame}
\frametitle{Study questions (example 2)}

\end{frame}

\section{Results}

\begin{frame}
\frametitle{Study participants overview}
\begin{itemize}
\item Have had the class that used Racket
\item Are not very familiar with Clojure 
\item Participants invited via department mailing list and old class lists (some additionally personally invited)
\item A study takes 1-1.5 hours, compensated for their time (to increase diversity)
\item So far: 11 participants: 10 male, 1 female; range from first year to seniors; CS majors and non-majors.  
\item Racket-based class: Fall 2015: 4, Fall 2014: 5, Earlier: 2. 
%\item Good at Racket: 2, were good at Racket, but forgot: 8, find Racket challenging: 1.
%\item Somewhat familiar with Clojure: 6, not familiar: 5.  
\end{itemize}
\end{frame}

\begin{frame}
\frametitle{Results}
%\emcomment{Results: questions solved and times}
{\bf Standard error messages} 
\vspace{0.1in}

\begin{tabular}{c | c| c| c | c }
\hline
{\bf ID} & {\bf Racket solved} & {\bf Racket time} & {\bf Clojure solved} & {\bf Clojure time} \\
\hline 
14 &  8 & ? &  6  &  21min \\
17 &  8 & 19min 58 sec &  7 &  21min \\
20* &  6 & 21min &  4 &  21min \\
23 &  7 & 21min &  6  &  21min \\
26 &  8 & ?? &  6 &  21min \\
\hline
\end{tabular}

*Participant 20 by mistake had two questions being the same in Racket and Clojure. 
\end{frame}

\begin{frame}
\frametitle{Results}
%\emcomment{Results: questions solved and times}
{\bf Modified (our) error messages}
\vspace{0.1in}

\begin{tabular}{c | c| c| c | c }
\hline
{\bf ID} & {\bf Racket solved} & {\bf Racket time} & {\bf Clojure solved} & {\bf Clojure time} \\
\hline 
2 & 7  & 20min 56sec & 8 & 16min 22sec  \\
6 &  8  & 14min 25sec &  4  &  21min \\
13 &  8 & 7min 53sec &  7 &  21min \\
16 &  8  & 14min 41sec &  8  &  15min 53sec \\
25 &  8  & 16min 46sec &  8  &  13min 9sec \\
29 &  4  & 21min &  2  &  21min \\
\hline
\end{tabular}
\end{frame}

\begin{frame}
\frametitle{Results}
%\emcomment{Results: questions solved and times}
{\bf Standard error messages: \Cl{background in Clojure}, \R{recent Racket}}
\vspace{0.1in}

\begin{tabular}{c | c| c| c | c }
\hline
{\bf ID} & {\bf Racket solved} & {\bf Racket time} & {\bf Clojure solved} & {\bf Clojure time} \\
\hline 
14 &  8 & ? &  \Cl{6}  &  \Cl{21min} \\
17 &  8 & 19min 58 sec &  7 &  21min \\
20* &  6 & 21min &  4 &  21min \\
23 &  7 & 21min &  6  &  21min \\
26 &  \R{8} & \R{??} &  \Cl{6} &  \Cl{21min} \\
\hline
\end{tabular}


%*Participant 20 by mistake had two questions being the same in Racket and Clojure. 
\end{frame}

\begin{frame}
\frametitle{Results}
%\emcomment{Results: questions solved and times}
{\bf Modified (our) error messages:  \Cl{background in Clojure}, \R{recent Racket}}
\vspace{0.1in}

\begin{tabular}{c | c| c| c | c }
\hline
{\bf ID} & {\bf Racket solved} & {\bf Racket time} & {\bf Clojure solved} & {\bf Clojure time} \\
\hline 
2 & 7  & 20min 56sec &  \Cl{8} & \Cl{16min 22sec} \\
6 &  8  & 14min 25sec &  4  &  21min \\
13 &  \R{8}  & \R{7min 53sec} &  \Cl{7}  &  \Cl{21min} \\
16 &  8  & 14min 41sec &  \Cl{8}  &  \Cl{15min 53sec} \\
25 &  8  & 16min 46sec &  \Cl{8}  &  \Cl{13min 9sec} \\
29 &  4  & 21min &  2  &  21min \\
\hline
\end{tabular}
\end{frame}

\begin{frame}
\frametitle{Results: being on track}
\emcomment{Results: being on track}
\end{frame}

\begin{frame}
\frametitle{Results:interview questions}
\emcomment{Interview questions and feedback}
\end{frame}

\section{Conclusions and future work}

\frametitle{Conclusions}
\begin{frame}
\emcomment{Our error messages vs standard}
\emcomment{Comparison to Racket}
\emcomment{Correlation to background}
\end{frame}

\begin{frame}
\frametitle{Conclusions}
\emcomment{Thoughts on study setup}
\end{frame}

\begin{frame}
\frametitle{Future work}
\emcomment{How much do people use error messages?}
\end{frame}

\begin{frame}
\frametitle{Acknowledgments}
	Our research was sponsored by:
	\begin{itemize}
	\item HHMI, UMN UROP, LSAMP
        \item Coginitect, Inc providing funding for participants' compensation 
	\end{itemize}
	{\centering
	\noindent
	Thank you! \par
	Any questions? \par
	}
\end{frame}
\end{document}
